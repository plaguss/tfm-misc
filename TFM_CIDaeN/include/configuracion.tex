% -------------------------------
% CIDaeN: Plantilla para la elaboración del TFM. Versión 0.0 
%
% Luis de la Ossa. UCLM
% 
% Compilar con XeLaTeX 
% -------------------------------


% -------------------------------
% Idioma
% -------------------------------
\usepackage{polyglossia}
\setmainlanguage{spanish}
\PassOptionsToPackage{spanish}{babel}
\usepackage{babel}

% -------------------------------
% Fuente
% -------------------------------
% Cualquier tamaño de texto: {\fontsize{100pt}{120pt}\selectfont tutexto}
\usepackage{anyfontsize}
% Selección de fuentes.
\usepackage{fontspec}
% Espacio entre líneas
\usepackage{setspace}
% Fuentes especiales
\usepackage{textcomp,marvosym,pifont}


% -------------------------------
% Otros paquetes de uso común
% -------------------------------
% Símbolos matemáticos
\usepackage{amsmath,amsfonts,amssymb}
% Gráficos
\usepackage{graphicx}
% Posición arbitraria de los gráficos
\usepackage[absolute,overlay]{textpos}
% Apéndices
\usepackage{appendix}


% -------------------------------
% Esquema de colores
% -------------------------------
% Paquete para la definición de colores
\usepackage[table]{xcolor}
% Esquema de colores
% -------------------------------
% Plantilla del TFG en la ESIIAB-UCLM. Versión 0.0 (Preliminar)
%
% Luis de la Ossa
% 
% Compilar con XeLaTeX 
% -------------------------------


% -------------------------------
% Esquema de colores
% -------------------------------
% Estos balores coinciden con colores de OpenOffice 
% Se pueden redefinir para que coincidan con las paletas de las 
% herramientas utilizadas para hacer las gráficas.
\definecolor{blanco}{RGB}{255,255,255} 
\definecolor{negro}{RGB}{0,0,0}
\definecolor{gris95}{gray}{.95}
\definecolor{gris75}{gray}{.75}
\definecolor{gris50}{gray}{.50}
\definecolor{gris25}{gray}{.25}
\definecolor{rojo}{RGB}{172,39,0}        
\definecolor{azul}{RGB}{2,87,108}        
\definecolor{verde}{RGB}{0,116,70}     
\definecolor{naranja}{RGB}{224, 119, 20} 
\definecolor{mostaza}{RGB}{226, 182, 21} 

% Negrita con color definido 
\newcommand{\bfa}[1]{\textcolor{azul}{\bf #1}} 
\newcommand{\bfr}[1]{\textcolor{rojo}{\bf #1}}
\newcommand{\bfn}[1]{\textcolor{naranja}{\bf #1}}
\newcommand{\bfv}[1]{\textcolor{verde}{\bf #1}}  
\newcommand{\bfm}[1]{\textcolor{mostaza}{\bf #1}} 



% -------------------------------
% Color del tema
% -------------------------------

% Color para los títulos, etc.
%\definecolor{tema}{RGB}{0,0,0}        % Negro 
\definecolor{tema}{RGB}{2,87,108}      % Azul del esquema  

% Negritas con el color del tema
\newcommand{\bft}[1]{\textcolor{tema}{\bf #1}} 



% -------------------------------
% Márgenes
% -------------------------------
% Márgenes de las páginas
\usepackage[
  inner	 =	3cm,   % Margen interior
  outer	 =	2.5cm, % Margen exterior
  top	 =	2.5cm, % Margen superior
  bottom =	2.5cm, % Margen inferior
  includeheadfoot, % Incluye cabecera y pie de página en los márgenes
]{geometry}
% Muestra una regla para comprobar el formato de las páginas (descomentar)
%\usepackage[type=upperleft,showframe,marklength=8mm]{fgruler} 


% -------------------------------
% Renombra tablas y bibliografía
% -------------------------------
\addto\captionsspanish{
	\renewcommand{\listtablename}{Índice de tablas} 
	\renewcommand{\tablename}{Tabla}
	\renewcommand{\bibname}{Referencia bibliográfica}
}


% -------------------------------
% Itemize y enumerate. Control de los espacios.
% -------------------------------
\usepackage{enumitem}
\setitemize{itemsep=0pt}
\setenumerate{itemsep=0pt}
\renewcommand{\labelitemi}{\tiny\bft{$\blacksquare$}}
\renewcommand{\labelitemii}{\bft{$\bullet$}}
\renewcommand{\labelitemiii}{\bft{-}}


% -------------------------------
% Enlaces y colores
% -------------------------------
\usepackage{hyperref}
%\hypersetup{
%    colorlinks,
%    citecolor=black,
%    filecolor=black,
%    linkcolor=black,
%    urlcolor=black
%}

% -------------------------------
% Gestión de elementos flotantes
% -------------------------------
% Para forzar a que las figuras aparezcan antes del fin de una sección.
%\usepackage[section]{placeins}

% Gestión de títulos de los elementos flotantes
\usepackage{caption}
% Subfiguras y títulos en subfiguras
\usepackage{subcaption}

% Títulos
\DeclareCaptionFont{tema}{\color{tema}} % Color
\captionsetup{labelfont={tema, bf}}     % Estilo
\captionsetup{font=normalsize}          % Tamaño
\captionsetup{width=.9\linewidth}       % Anchura del título
 % Distancia de la figura al texto.
\setlength{\intextsep}{1cm} 
\setlength{\textfloatsep}{1cm}         
       

% -------------------------------
% Utilidades para tablas
% -------------------------------
% Para fundir filas en tablas
\usepackage{multirow}
% Para definir columnas con ancho fijo y alineación
\usepackage{array,ragged2e}
\newcolumntype{L}[1]{>{\raggedright\let\newline\\\arraybackslash\hspace{0pt}}m{#1}}
\newcolumntype{C}[1]{>{\centering\let\newline\\\arraybackslash\hspace{0pt}}m{#1}}
\newcolumntype{R}[1]{>{\raggedleft\let\newline\\\arraybackslash\hspace{0pt}}m{#1}}


% -------------------------------
% Para tabular
% -------------------------------
\usepackage{tabto}


% -------------------------------
% Formato de títulos, capítulos, etc.
% -------------------------------
\usepackage{titlesec, titletoc}
% Parte
\titleformat{\part}[display]												   	   % Estilo
	{\Huge \centering \color{tema}}                                                % Formato
	{Parte \thepart }                                                              % Etiqueta
	{0.75ex}                                                                       % Separación
	{\bfseries\fontsize{25pt}{40pt}\selectfont}                                    % Entre etiqueta y título            
	[\thispagestyle{empty}]                                                        % Después

% Capítulo		
\titleformat{\chapter}[block]            			 			                    % Estilo  	
	{\vspace{0cm} \flushright \bfseries \LARGE \color{tema}}  	        			% Formato 
	{\thechapter.}                  			  									% Etiqueta
	{0.75ex}                                     									% Separación
	{}                                           									% Entre etiqueta y título
	[\vspace{0.1cm} \rule{0.5\textwidth}{1pt}\vspace*{1cm}\thispagestyle{plain}]    % Después      
  
% Sección
\titleformat{\section}
	{\vspace{5pt}  \Large \color{tema}}
	{\thesection .}
	{0.75ex}
	{}

% Subsección
\titleformat{\subsection}
	{\large \color{tema}}
	{\thesubsection .}
	{0.75ex}
	{}
 
% Subsubsección
\titleformat{\subsubsection}
	{\bf\color{tema}}
	{}
	{0.75ex}
	{}
	 
% Parte (TOC)
\titlecontents{part}[2pc]
  	{\vspace{20pt}\bfseries\filright\Large\color{tema}}
  	{\contentslabel{1.5pc}}{\hspace*{-1.5pc}}
  	{\vspace{5pt}}
  	{}
  
% Capítulo (TOC)
\titlecontents{chapter}[2pc]
  	{\vspace{10pt}\bfseries\large\filright}
  	{\contentslabel{1.5pc}}{\hspace*{-1.5pc}}
  	{\mdseries\titlerule*[0.5pc]{.}\bfseries\contentspage}

% Sección (TOC)
\titlecontents{section}[4pc]
  	{\vspace{5pt}\filright}
  	{\contentslabel{2pc}}{\hspace*{2pc}}
  	{\titlerule*[0.5pc]{.}\contentspage}
  	{\vspace{5pt}

% Subsección (TOC)
\titlecontents{subsection}[6pc]
  	{\vspace{2.5pt}\filright\small\em}
  	{\contentslabel{3pc}}{\hspace*{-3pc}}
  	{\titlerule*[0.5pc]{.}\contentspage} 
  	
  	
% -------------------------------
% Cabeceras y pie de página
% -------------------------------
\usepackage{fancyhdr}
\pagestyle{fancy}
\fancyhf{}

% Permite incluir la parte. Para ello crea \parttitle
\let\Oldpart\part
\newcommand{\parttitle}{}
\renewcommand{\part}[1]{\Oldpart{#1}\renewcommand{\parttitle}{#1}}

% Cabeceras y pies de página
\fancyhead[LE]{}
\fancyhead[RO]{\slshape \leftmark}
\fancyfoot[LE,RO]{\thepage}
\renewcommand{\footrulewidth}{1pt}
\renewcommand{\headrulewidth}{1pt}
\renewcommand{\chaptermark}[1]{\markboth{#1}{}} % Capítulo (Solo texto)
\renewcommand{\chaptermark}[1]{\markboth{\thechapter. #1}{}} % Capítulo (Número y texto)

% Formato plano para las páginas con títulos (solo incluye el pie de página)
\fancypagestyle{plain}{
\fancyhf{}
\fancyhead{}
\fancyfoot[LE,RO]{\thepage}
\renewcommand{\footrulewidth}{1pt}
\renewcommand{\headrulewidth}{0pt}
}

 
% -------------------------------
% Utilidades
% -------------------------------
% Marcas
\newcommand{\pcite}{\bfr{\large[?]}\,} % Cita pendiente: [?] en rojo y negrita.
\newcommand{\ps}{\bfr{\huge[--}\,} % Pricipio de un bloque en sucio
\newcommand{\fs}{\bfr{\huge--]}\,} % Fin de un bloque en sucio.

% Notas
\usepackage[textsize=tiny,spanish,shadow,textwidth=2cm]{todonotes}

% Para generar textos "random"
\usepackage{lipsum} 
 

% -------------------------------
% Algoritmos y código
% -------------------------------

% Se definen entornos específicos para algoritmos y código que permiten
% gestionar los títulos manera similar en ambos casos, y similar al resto
% de elementos flotantes.
\usepackage{newfloat}
\usepackage{float}
\DeclareFloatingEnvironment[ % Algoritmos
  listname = {Índice de algoritmos},
  name = Algoritmo
]{algorithm}

\DeclareFloatingEnvironment[ % Fragmentos de código
  listname = {Índice de listados de código},
  name = Código
]{code}

% Define un estilo para el encabezado de algoritmos y código.
\DeclareCaptionFormat{algcode}{
% OPCIÓN 1 -> Línea debajo del título
\rule{\dimexpr\textwidth\relax}{0.4pt}\vspace{0cm}
#1#2#3
\vspace{-0.2cm}\rule{\dimexpr\textwidth\relax}{0.4pt} % 
% OPCIÓN 2 -> Sin línea debajo del título
%\rule{\dimexpr\textwidth\relax}{0.4pt}\vspace{0cm}
%#1#2#3
}
\captionsetup[algorithm]{format=algcode, width=\linewidth}
\captionsetup[code]{format=algcode, width=\linewidth}

% Opciones específicas para algoritmos.
\usepackage[noend]{algpseudocode}
% Tamaño de letra y separadores
\makeatletter
\renewcommand{\ALG@beginalgorithmic}{\small\hrule\vskip10pt}
\makeatother
% Variables y funciones
\newcommand{\va}[1]{{\it {#1}}}				
\newcommand{\fu}[1]{{\textsc{#1}}}          
% Palabras clave
\renewcommand{\algorithmicrepeat}{\bft{repetir}}
\renewcommand{\algorithmicuntil}{\bft{hasta}}
\renewcommand{\algorithmicend}{\bft{fin}}
\renewcommand{\algorithmicif}{\bft{si}}
\renewcommand{\algorithmicthen}{\bft{entonces}}
\renewcommand{\algorithmicelse}{\bft{si no}}
\newcommand{\algorithmicelsif}{\algorithmicelse\ \algorithmicif}
\newcommand{\algorithmicendif}{\algorithmicend\ \algorithmicif}
\renewcommand{\algorithmicfor}{\bft{para}}
\renewcommand{\algorithmicforall}{\bft{para cada}}
\renewcommand{\algorithmicwhile}{\bft{mientras}}
\renewcommand{\algorithmicdo}{\bft{hacer}}
\renewcommand{\algorithmicprocedure}{\bft{procedimiento}}
\renewcommand{\algorithmicreturn}{\bft{devolver}}
\renewcommand{\algorithmicfunction}{\bft{función}}
\renewcommand{\algorithmicloop}{\bft{iterar}}

% Opciones específicas para código.
\usepackage{listings}
% Aspecto de los listados de código (ejemplo)
% Los colores se han de cambiar y adaptar a los distintos lenguajes de programación.
\lstset{
    language=Python,
    keywordstyle=\bf\color{tema},  
	breaklines=true,    
	basicstyle={\small\ttfamily},
    tabsize=5,
    frame=t % Es importante no cambiar esta opción (es la línea superior) 
}

% -------------------------------
% Cacacteres pifont de uso común (con color)
% -------------------------------
\newcommand{\vmarkt}{\textcolor{tema}{\ding{52}} }  % Checked (V) Tema
\newcommand{\vmarkr}{\textcolor{rojo}{\ding{52}} }  % Checked (V) Rojo
\newcommand{\vmarka}{\textcolor{azul}{\ding{52}} }  % Checked (V) Azul

\newcommand{\xmarkt}{\textcolor{tema}{\ding{55}} }  % Checked (X) Tema
\newcommand{\xmarkr}{\textcolor{rojo}{\ding{55}} }  % Checked (X) Rojo
\newcommand{\xmarka}{\textcolor{azul}{\ding{55}} }  % Checked (X) Azul

\newcommand{\lmarkt}{\textcolor{tema}{\ding{46}} }  % Lápiz Tema
\newcommand{\lmarkr}{\textcolor{rojo}{\ding{46}} }  % Lápiz Rojo
\newcommand{\lmarka}{\textcolor{azul}{\ding{46}} }  % Lápiz Azul

\newcommand{\hmarkt}{\textcolor{tema}{\ding{43}} }  % Mano Tema
\newcommand{\hmarkr}{\textcolor{rojo}{\ding{43}} }  % Mano Roja 
\newcommand{\hmarka}{\textcolor{azul}{\ding{43}} }  % Mano Azula 

\newcommand{\fmarkt}{\textcolor{tema}{\ding{220}} }  % Flechza Tema
\newcommand{\fmarkr}{\textcolor{rojo}{\ding{220}} }  % Flecha Roja 
\newcommand{\fmarka}{\textcolor{azul}{\ding{220}} }  % Flecha Azul