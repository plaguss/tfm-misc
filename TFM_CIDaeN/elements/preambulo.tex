% -------------------------------
% CIDaeN: Plantilla para la elaboración del TFM. Versión 0.0 
%
% Luis de la Ossa. UCLM
% 
% Compilar con XeLaTeX 
% -------------------------------



% -------------------------------
% Dedicatoria
% -------------------------------
%\cleardoublepage
%\clearpage
%\thispagestyle{empty}

%\vspace*{9cm}  
%\begin{flushright} \em 
%Aquí va la dedicatoria que cada cual \\ 
%quiera escribir. El ancho se controla \\ 
%manualmente
%\end{flushright}

% -------------------------------
% Declaración de autoría
% -------------------------------

%\cleardoublepage
\clearpage
\thispagestyle{plain}
\setcounter{page}{1} \null
\begin{center}
\Large{\bft{Declaración de autoría}}
\end{center}
\vskip1cm

Yo, Agustín Piqueres Lajarín, con DNI 48675488J, declaro que soy el único autor del trabajo fin de grado titulado ``Clasificación de movimientos de CrossFit:
una aplicación con MoViNets'', que el citado trabajo no infringe las leyes en vigor sobre propiedad intelectual, y que todo el material no original contenido en dicho trabajo está apropiadamente atribuido a sus legítimos autores.

\vspace*{2cm}
\begin{center}
Albacete, a 12 de Septiembre de 2022

\vskip3cm

Fdo.: \autor
\end{center}


% -------------------------------
% Resumen
% -------------------------------
%\cleardoublepage
\clearpage
\thispagestyle{plain}
\begin{center}
\Large{\bft{Resumen}}
\end{center}
\vskip1cm

El CrossFit como deporte como deporte competitivo ha venido ganando adeptos en los últimos años, y el primer paso de clasificación en una competición consiste en realizar una o varias pruebas (\textit{workouts}) que deben ser grabadas y enviadas a la organización para su revisión.

En este trabajo se desarrolla una aplicación para clasificar movimientos de CrossFit como primer paso hacia una aplicación que permita automatizar la revisión de las pruebas enviadas por los atletas. Para llevar a cabo el mismo se crea un conjunto de \textit{clips} de movimientos, basado en videos realizados por competidores de élite de CrossFit perteneciente a la final de los Crossfit Games 2020. El clasificador utilizado se basa en \textit{MoViNets}, una familia de redes neuronales para videos eficiente tanto en términos de computación como de memoria, y se presenta una aplicación desarrollada en AWS. Todo el código está disponible públicamente, y una pequeña librería utilizada para trabajar con \textit{MoViNets} está disponible en \href{https://pypi.org/project/movinets_helper/}{PyPI}.



% -------------------------------
% Agradecimientos
% -------------------------------
%\cleardoublepage
%\clearpage
%\thispagestyle{plain}
%\begin{center}
%\Large{\bft{Agradecimientos}}
%\end{center}
%\vskip1cm

%AGRADECIMIENTOS AQUÍ
