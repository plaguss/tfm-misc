\chapter{Marcas y ayudas}

Se han introducido algunos elementos que permiten dejar marcas para indicar que algunas cosas provisionales se han de corregir. Por ejemplo \verb+\pcite+ deja esta señal \pcite para indicar que se ha de añadir una cita bibliográfica en un punto concreto. \\

\ps

Las marcas \verb+\ps+ y \verb+\fs+ permiten marcar y delimitar un bloque con contenido provisional, y que debe ser revisado posteriormente.

\fs \\


Existen métodos más avanzados para hacer anotaciones, como los que ofrece \verb+todonotes+. Permiten insertar notas al margen \todo{Como por ejemplo ésta.}.  \\

También  permite utilizar notas en el propio texto.  \todo[inline]{En internet hay combinaciones de colores interesantes para destacar distintos tipos de notas}  


\section{Figuras}

Este paquete tiene otros comandos como \verb+\missingfigure+ para indicar que faltan figuras. \\

\missingfigure{Aquí se puede escribir la descripción de la figura que falta}